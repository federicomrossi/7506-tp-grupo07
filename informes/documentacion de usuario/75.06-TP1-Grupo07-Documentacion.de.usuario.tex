\documentclass{article}

%% PAQUETES

% Paquetes generales
\usepackage[margin=2cm, paperwidth=210mm, paperheight=297mm]{geometry}
\usepackage[spanish]{babel}
\usepackage[utf8]{inputenc}
\usepackage{gensymb}

% Paquetes para estilos
\usepackage{textcomp}
\usepackage{setspace}
\usepackage{colortbl}
\usepackage{color}
\usepackage{color}
\usepackage{upquote}
\usepackage{xcolor}
\usepackage{listings}
\usepackage{caption}
\usepackage[T1]{fontenc}
\usepackage[scaled]{beramono}

% Paquetes extras
\usepackage{amssymb}
\usepackage{float}
\usepackage{graphicx}
\usepackage{url}
\usepackage{color}


%% Fin PAQUETES


% Definición de preferencias para la impresión de código fuente.
%% Colores
\definecolor{gray99}{gray}{.99}
\definecolor{gray95}{gray}{.95}
\definecolor{gray75}{gray}{.75}
\definecolor{gray50}{gray}{.50}
\definecolor{gray25}{gray}{.25}
\definecolor{keywords_blue}{rgb}{0.13,0.13,1}
\definecolor{comments_green}{rgb}{0,0.5,0}
\definecolor{strings_red}{rgb}{0.9,0,0}

%% Caja de código
\DeclareCaptionFont{white}{\color{white}}
\DeclareCaptionFont{style_labelfont}{\color{black}\textbf}
\DeclareCaptionFont{style_textfont}{\it\color{black}}
\DeclareCaptionFormat{listing}{\colorbox{gray95}{\parbox{16.78cm}{#1#2#3}}}
\captionsetup[lstlisting]{format=listing,labelfont=style_labelfont,textfont=style_textfont}

\lstset{
	aboveskip = {1.5\baselineskip},
	backgroundcolor = \color{gray99},
	basicstyle = \ttfamily\footnotesize,
	breakatwhitespace = true,   
	breaklines = true,
	captionpos = t,
	columns = fixed,
	commentstyle = \color{comments_green},
	escapeinside = {\%*}{*)}, 
	extendedchars = true,
	frame = lines,
	keywordstyle = \color{keywords_blue}\bfseries,
	language = Oz,                       
	numbers = left,
	numbersep = 5pt,
	numberstyle = \tiny\ttfamily\color{gray50},
	prebreak = \raisebox{0ex}[0ex][0ex]{\ensuremath{\hookleftarrow}},
	rulecolor = \color{gray75},
	showspaces = false,
	showstringspaces = false, 
	showtabs = false,
	stepnumber = 1,
	stringstyle = \color{strings_red},                                    
	tabsize = 2,
	title = \null, % Default value: title=\lstname
	upquote = true,                  
}

%% FIGURAS
\captionsetup[figure]{labelfont=bf,textfont=it}
%% TABLAS
\captionsetup[table]{labelfont=bf,textfont=it}

% COMANDOS

%% Titulo de las cajas de código
\renewcommand{\lstlistingname}{Código}
%% Titulo de las figuras
\renewcommand{\figurename}{Figura}
%% Titulo de las tablas
\renewcommand{\tablename}{Tabla}
%% Referencia a los códigos
\newcommand{\refcode}[1]{\textit{Código \ref{#1}}}
%% Referencia a las imagenes
\newcommand{\refimage}[1]{\textit{Imagen \ref{#1}}}


\begin{document}


% TÍTULO, AUTORES Y FECHA
\begin{titlepage}
	\vspace*{\fill}
	\begin{center}
		\Large 75.06 Organización de Datos \\
		\Huge TP N°1: Catálogo discográfico \\
		\bigskip\huge\textit{Grupo 07} \\
		\bigskip\bigskip\bigskip\bigskip\bigskip\bigskip
		\bigskip\bigskip\bigskip\bigskip\bigskip\bigskip\bigskip
		\medskip\huge\textit{``Documentación de usuario''} \\
		\date{}
	\end{center}
	\vspace*{\fill}
\end{titlepage}
\newpage



% ÍNDICE
\tableofcontents
\newpage




% INTRODUCCIÓN
\section{Introducción}
	
	El siguiente documento tiene como objetivo proporcionar al usuario toda la información necesaria para la correcta instalación y utilización del software para el armado de un catálogo musical.
	\par
	Este documento va desde los recaudos a tomar previos a la compilación hasta la forma de utilización de la herramienta en si.
\bigskip




% REQUERIMIENTOS DEL SISTEMA
\section{Requerimientos del sistema}

	Antes de instalar y configurar el programa, asegúrese de que su sistema cumple los requisitos mínimos recomendados, los cuales se encuentran especificados a continuación:
	\medskip

\begin{itemize}
\itemsep=5pt \topsep=0pt \partopsep=0pt \parskip=0pt \parsep=0pt

	\item \textit{Sistemas operativos}: GNU/Linux (x86 y x86-64, distribuciones Linux basadas en RPM y DEB);

	\item \textit{Controlador de versions}: GIT (\url{http://git-scm.com/});

	\item \textit{Compilador}: g++ (\url{http://gcc.gnu.org/});

	\item \textit{Herramientas}: Make (\url{http://www.gnu.org/software/make/}).

\end{itemize}
\bigskip




% DESCARGA DEL SOFTWARE
\section{Descarga del software}

	Para obtener la ultima versión de este software, dirijase a la línea de comandos (terminal) de su sistema operativo e ingrese el siguiente comando:
	\bigskip

	\colorbox{gray95}{{\ttfamily\footnotesize
	\$ git clone \url{git://github.com/federicomrossi/7506-tp-grupo07.git}\\}}
	\bigskip

	Esto descargará los archivos almacenados en el repositorio, ubicándose en la carpáta \textit{código} los archivos fuentes que hacen a la implementación del programa.
\bigskip\medskip




% PASOS PREVIOS A LA UTILIZACION
\section{Pasos previos a la utilización}

	Antes de poder hacer uso de la aplicación es necesario realizar unas pocas tareas previas para la puesta a punto de la misma. Las mismas serán detalladas en lo que sigue de esta sección.
	\bigskip



% PASOS PREVIOS A LA UTILIZACION - Configuracion del entorno
\subsection{Configuración del entorno}

	Previo a la compilación de la herramienta deberán configurase ciertos parámetros. Los mismos se encuentran en el archivo “<PONER ARCHIVO> DEFINITIVO>” de subdirectorio <PONER SUBDIRECTORIO DEFINITIVO>.
	\par
	Es condición necesaria la existencia de los subdirectorios configurados ya que la herramienta no se encarga de crearlos por si misma. Las rutas y parámetros que deben especificarse son:
	\medskip


\begin{itemize}
\itemsep=5pt \topsep=0pt \partopsep=0pt \parskip=0pt \parsep=0pt

	\item \textit{SOURCE\_PATH}: Directorio donde se encuentran los temas a indexar;

	\item \textit{DEST\_PATH}: Directorio donde se escribirán los archivos del sistema;

	\item \textit{MAX\_BLOCK\_SIZE} [REVISAR]: Tamaño de los bloques en los archivos;

	\item \textit{MAX\_REGS\_PER\_BLOCK} [REVISAR]: Cantidad máxima de registros por bloque.

\end{itemize}
\bigskip



% PASOS PREVIOS A LA UTILIZACION - Compilacion
\subsection{Compilación}

	Para la compilación simplemente abra una terminal, diríjase hacia el directorio en donde se realizo la clonación del software y sitúese en el subdirectorio \textit{código}. Luego, ejecute el siguiente comando:
	\bigskip

	\colorbox{gray95}{{\ttfamily\footnotesize
	\$ make\\}}
	\medskip

	Esto dará comienzo al proceso de compilación de los archivos fuente.
\bigskip\medskip




% EJECUCION
\section{Ejecución}

	Una vez completada la compilación, se podrá correr el programa mediante el comando que sigue (recordar que la aplicación debe ser utilizada desde la misma terminal de línea de comados):
	\bigskip

	\colorbox{gray95}{{\ttfamily\footnotesize
	\$ make run\\}}
	\medskip



% UTILIZACION DE LA INTERFAZ DEL PROGRAMA
\section{Utilización de la interfaz del programa}

	Al iniciar el programa se puede observar un menú principal como el que se muestra en la \textit{Figura 1}. En este se listan múltiples opciones, las cuales seran detalladas el los siguientes apartados.
	\bigskip

	[ COLOCAR IMAGEN ]
	\bigskip



% UTILIZACION DE LA INTERFAZ DEL PROGRAMA - Indexacion de canciones
\subsection{Indexación de canciones}

	[ COLOCAR TEXTO AQUI ]
\bigskip



% UTILIZACION DE LA INTERFAZ DEL PROGRAMA - Indexacion en modo append
\subsection{Indexación de canciones en modo append}

	[ COLOCAR TEXTO AQUI ]
\bigskip



% UTILIZACION DE LA INTERFAZ DEL PROGRAMA - Busqueda de canciones por autor
\subsection{Búsqueda de canciones por autor}

	[ COLOCAR TEXTO AQUI ]
\bigskip



% UTILIZACION DE LA INTERFAZ DEL PROGRAMA - Busqueda de canciones por titulo
\subsection{Búsqueda de canciones por titulo}

	[ COLOCAR TEXTO AQUI ]
\bigskip



% UTILIZACION DE LA INTERFAZ DEL PROGRAMA - Busqueda de canciones por frase
\subsection{Búsqueda de canciones por frase}

	[ COLOCAR TEXTO AQUI ]
\bigskip



\end{document}
