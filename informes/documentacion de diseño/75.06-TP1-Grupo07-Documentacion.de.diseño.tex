\documentclass{article}

%% PAQUETES

% Paquetes generales
\usepackage[margin=2cm, paperwidth=210mm, paperheight=297mm]{geometry}
\usepackage[spanish]{babel}
\usepackage[utf8]{inputenc}
\usepackage{gensymb}

% Paquetes para estilos
\usepackage{textcomp}
\usepackage{setspace}
\usepackage{colortbl}
\usepackage{color}
\usepackage{color}
\usepackage{upquote}
\usepackage{xcolor}
\usepackage{listings}
\usepackage{caption}
\usepackage[T1]{fontenc}
\usepackage[scaled]{beramono}

% Paquetes extras
\usepackage{amssymb}
\usepackage{float}
\usepackage{graphicx}
\usepackage{url}

%% Fin PAQUETES


% Definición de preferencias para la impresión de código fuente.
%% Colores
\definecolor{gray99}{gray}{.99}
\definecolor{gray95}{gray}{.95}
\definecolor{gray75}{gray}{.75}
\definecolor{gray50}{gray}{.50}
\definecolor{keywords_blue}{rgb}{0.13,0.13,1}
\definecolor{comments_green}{rgb}{0,0.5,0}
\definecolor{strings_red}{rgb}{0.9,0,0}

%% Caja de código
\DeclareCaptionFont{white}{\color{white}}
\DeclareCaptionFont{style_labelfont}{\color{black}\textbf}
\DeclareCaptionFont{style_textfont}{\it\color{black}}
\DeclareCaptionFormat{listing}{\colorbox{gray95}{\parbox{16.78cm}{#1#2#3}}}
\captionsetup[lstlisting]{format=listing,labelfont=style_labelfont,textfont=style_textfont}

\lstset{
	aboveskip = {1.5\baselineskip},
	backgroundcolor = \color{gray99},
	basicstyle = \ttfamily\footnotesize,
	breakatwhitespace = true,   
	breaklines = true,
	captionpos = t,
	columns = fixed,
	commentstyle = \color{comments_green},
	escapeinside = {\%*}{*)}, 
	extendedchars = true,
	frame = lines,
	keywordstyle = \color{keywords_blue}\bfseries,
	language = Oz,                       
	numbers = left,
	numbersep = 5pt,
	numberstyle = \tiny\ttfamily\color{gray50},
	prebreak = \raisebox{0ex}[0ex][0ex]{\ensuremath{\hookleftarrow}},
	rulecolor = \color{gray75},
	showspaces = false,
	showstringspaces = false, 
	showtabs = false,
	stepnumber = 1,
	stringstyle = \color{strings_red},                                    
	tabsize = 2,
	title = \null, % Default value: title=\lstname
	upquote = true,                  
}

%% FIGURAS
\captionsetup[figure]{labelfont=bf,textfont=it}
%% TABLAS
\captionsetup[table]{labelfont=bf,textfont=it}

% COMANDOS

%% Titulo de las cajas de código
\renewcommand{\lstlistingname}{Código}
%% Titulo de las figuras
\renewcommand{\figurename}{Figura}
%% Titulo de las tablas
\renewcommand{\tablename}{Tabla}
%% Referencia a los códigos
\newcommand{\refcode}[1]{\textit{Código \ref{#1}}}
%% Referencia a las imagenes
\newcommand{\refimage}[1]{\textit{Imagen \ref{#1}}}


\begin{document}


% TÍTULO, AUTORES Y FECHA
\begin{titlepage}
	\vspace*{\fill}
	\begin{center}
		\Large 75.06 Organización de Datos \\
		\Huge TP N°1: Catálogo discográfico \\
		\bigskip\huge\textit{Grupo 07} \\
		\bigskip\bigskip\bigskip\bigskip\bigskip\bigskip
		\bigskip\bigskip\bigskip\bigskip\bigskip\bigskip\bigskip
		\medskip\huge\textit{``Documentación de diseño''} \\
		\date{}
	\end{center}
	\vspace*{\fill}
\end{titlepage}
\newpage



% ÍNDICE
\tableofcontents
\newpage


% INTRODUCCIÓN
\section{Introducción}
	
	El siguiente documento tiene como objetivo especificar y precisar los detalles técnicos de la realización de la primera parte del trabajo práctico. Se explican los problemas surgidos a la hora del diseño y las soluciones encaradas para solventar los mismos.
	\par
	Todos los archivos y códigos fuente aquí mencionados y relacionados al proyecto, así como también el presente informe, pueden ser encontrados y descargados del repositorio del grupo (\url{https://github.com/federicomrossi/7506-tp-grupo07}).
\bigskip




% REQUERIMIENTOS PLANTEADOS
\section{Requerimientos planteados}

	El trabajo solicitado se basa en la creación de un sistema para la generación de un catalogo a partir de archivos de letras de canciones de diversos autores. Dicho sistema debe ser capaz de realizar búsquedas por autor, titulo y frases, entregando como resultado los temas que coincidan con los parámetros ingresados.
	\par
	A su vez se exigieron ciertas condiciones para la realización del trabajo a la hora del manejo de las estructuras. El índice para la búsqueda por titulo debía estructurarse como un hash, por autor como un árbol B+ y los archivos de texto como bloques de registros variables.
	\par
	El producto final es una aplicación de consola que permite realizar las operaciones pedidas.



% DIVISION DEL TRABAJO
\section{División del trabajo}

	El trabajo esta divido en tres capas funcionales: la física, la lógica y el front-end.
	\par
	A su vez, se la capa lógica esta formada por tres módulos principales, a saber, el árbol B+, el hash extensible y las estructuras de recuperación de textos.

\end{document}
